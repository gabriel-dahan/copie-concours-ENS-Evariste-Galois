\documentclass[11pt]{article}

\usepackage{graphicx}
\setkeys{Gin}{keepaspectratio}
\let\Oldincludegraphics\includegraphics

\usepackage{caption}
\DeclareCaptionFormat{nocaption}{}
\captionsetup{format=nocaption,aboveskip=0pt,belowskip=0pt}

\usepackage{float}
\floatplacement{figure}{H} % forces figures to be placed at the correct location
\usepackage{xcolor} % Allow colors to be defined
\usepackage{enumerate} % Needed for markdown enumerations to work
\usepackage[margin=2cm]{geometry} % Used to adjust the document margins
\usepackage{amsmath} % Equations
\usepackage{amssymb} % Equations
\usepackage{textcomp} % defines textquotesingle

\usepackage{hyperref}

    
\title{CONCOURS GENERAL DE L'AN 1828}
\author{Evariste Galois \\ \textit{Transcrit par Gabriel Dahan}}
\date{}

\begin{document}
    
\maketitle

\textit{Ce document est une transcription en \LaTeX{} de la copie du concours g\'en\'eral de l'an 1828/9 (à l'ENS d'Ulm), 
r\'edig\'ee par Evariste Galois, disponible ici : \url{https://math.univ-lyon1.fr/~caldero/galois_copie.pdf}.
Les mots que je n'ai pas pu d\'echiffrer sont remplac\'es par des `\dots`.}\newline
\newline
    
Soit $Ex=0$ l'équation pour laquelle en \dots la limite supérieure $K$ des racines, \dots dans
laquelle nous supposerons pour plus de simplicité le plus haut terme positif. Comme 
l'hypothèse $x=+\infty$ donne pour résultat $Ex>0$, et qu'aucune racine ne doit être 
comprise entre $+\infty$ et une limite supérieure des racines, il faudrait que toute limite 
supérieure des racines substituée dans l'équation doit donner un résultat positif. Mais 
$K$ étant une limite, $K+z$ ($z$ étant postitif) en est encore une. Donc $E(K+z)$ doit 
être positif pour toute valeur positive de $z$, $K$ sera limite. Car aucune valeur \dots supérieure à
$K$ n'annulera $Ex$.

Il faut donc et il suffit que pour toute valeur positive de $z$, on ait $E(K+z)$ ou bien 

$$Ek+E'k\cdot z+\frac{1}{2}E''k\cdot z^2+\frac{1}{2\cdot 3}E'''k\cdot z^3+\dots$$

positif. Et cette condition sera évidemment remplie, si l'on suppose que tous les coefficients 
de $z$ dans cette fonction soient positifs.

Ainsi, on n'a qu'à résoudre le système d'inégalités 

$$Ek>0, \quad E'k>0, \quad E''k>0, \quad \dots, \quad E^{(m-1)}k>0$$

Pour celà, on cherchera le plus petit nombre entier qui satisfasse à la dernière, puis
le plus petit nombre qui satisfasse à la fois aux deux dernières, puis aux trois 
dernières, et ainsi de suite jusqu'à ce qu'on ait le plus petit nombre qui rend tous
les termes positifs. Arrivé à ce nombre, on aura la limite supérieure cherchée.

Si l'on 

\end{document}
