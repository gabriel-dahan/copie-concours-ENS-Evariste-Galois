\documentclass[11pt]{article}

\usepackage{graphicx}
\setkeys{Gin}{keepaspectratio}
\let\Oldincludegraphics\includegraphics

\usepackage{caption}
\DeclareCaptionFormat{nocaption}{}
\captionsetup{format=nocaption,aboveskip=0pt,belowskip=0pt}

\usepackage{float}
\floatplacement{figure}{H} % forces figures to be placed at the correct location
\usepackage{xcolor} % Allow colors to be defined
\usepackage{enumerate} % Needed for markdown enumerations to work
\usepackage[margin=2cm]{geometry} % Used to adjust the document margins
\usepackage{amsmath} % Equations
\usepackage{amssymb} % Equations
\usepackage{textcomp} % defines textquotesingle

\usepackage{hyperref}

    
\title{CONCOURS GENERAL DE L'AN 1828}
\author{Evariste Galois \\ \textit{Transcrit par Gabriel Dahan}}
\date{}

\begin{document}
    
\maketitle

\textit{Ce document est une transcription en \LaTeX{} de la copie du concours g\'en\'eral de l'an 1828/9 (à l'ENS d'Ulm), 
r\'edig\'ee par Evariste Galois, disponible ici : \url{https://math.univ-lyon1.fr/~caldero/galois_copie.pdf}.
Les mots que je n'ai pas pu d\'echiffrer sont remplac\'es par des `\dots`.}\newline
\newline
    
\quad\quad{\textit{$1^{ere}$ Question\quad $1^{ere}$ Méthode}}\newline

Soit $Ex=0$ l'équation pour laquelle en \dots la limite supérieure $K$ des racines, \dots dans
laquelle nous supposerons pour plus de simplicité le plus haut terme positif. Comme 
l'hypothèse $x=+\infty$ donne pour résultat $Ex>0$, et qu'aucune racine ne doit être 
comprise entre $+\infty$ et une limite supérieure des racines, il faudrait que toute limite 
supérieure des racines substituée dans l'équation doit donner un résultat positif. Mais 
$K$ étant une limite, $K+z$ ($z$ étant postitif) en est encore une. Donc $E(K+z)$ doit 
être positif pour toute valeur positive de $z$, $K$ sera limite. Car aucune valeur \dots supérieure à
$K$ n'annulera $Ex$.

Il faut donc et il suffit que pour toute valeur positive de $z$, on ait $E(K+z)$ ou bien 

$$Ek+E'k\cdot z+\frac{1}{2}E''k\cdot z^2+\frac{1}{2\cdot 3}E'''k\cdot z^3+\dots$$

positif. Et cette condition sera évidemment remplie, si l'on suppose que tous les coefficients 
de $z$ dans cette fonction soient positifs.

Ainsi, on n'a qu'à résoudre le système d'inégalités 

$$Ek>0, \quad E'k>0, \quad E''k>0, \quad \dots, \quad E^{(m-1)}k>0$$

Pour celà, on cherchera le plus petit nombre entier qui satisfasse à la dernière, puis
le plus petit nombre qui satisfasse à la fois aux deux dernières, puis aux trois 
dernières, et ainsi de suite jusqu'à ce qu'on ait le plus petit nombre qui rend tous
les termes positifs. Arrivé à ce nombre, on aura la limite supérieure cherchée.

Si l'on demandait au contraire la limite inférieure $l$ des racines, on ferait dans
$Ex$, $x=-y$, on chercherait la limite supérieure $K$ des racines de l'équation
$E(-y)=0$, et l'on ferait $l=-K$, et comme $K$ est plus grand que toutes les valeurs de
$y$, il s'ensuit que $-K$ ou $l$ est plus petit que toutes celles de $-y$ ou de $x$.

On peut présenter cette règle, appelée Méthode de Newton, d'une manière un peu plus simple.

Puisque $K$ est plus grand et $l$ plus petit que toutes les racines de l'équation $Ex=0$,
il faudra que l'équation en $z$, $E(K+z)=0$ n'ait pas de racine $>0$, sans quoi 
$Ex=0$ aurait des racines $>K$, et de même, que l'équation en $z$, $E(l+z)=0$ 
n'ait pas de racine $<0$, sans quoi $Ex=0$ aurait des racines $<l$. Les condition à
exprimer sont donc que $E(K+z)=0$ n'ait pas de racine positive, et que $E(l+z)=0$
n'en ait pas de négative. Il suffit pour celà que la première n'ait que des
\dots, et la seconde, que des variations de signe. C'est qui donnera encore
deux systèmes d'inégalités à résoudre dont l'un donnera $K$ et l'autre $l$.\newline

\quad\quad{\textit{$2^{eme}$ Méthode}}\newline

La $1^{ere}$ Méthode donne en général des approximations assez bonnes, mais on voit
qu'elle est longue et pénible dans la pratique. En voici une plus expéditive.

Soit $m$ le degré de l'équation, $n+1$ le rang du premier terme négatif, 
$N$ le plus grand des coefficients des termes négatifs, pris positivement, en sorte que
l'équation débarrassée du coefficient fractionnaire, soit de la forme:

\begin{equation}
x^m+\dots+Hx^{m-n+1}-Kx^{m-n}-\dots -Nx^p\dots=Ex=0
\end{equation}

Il s'agit d'abord de trouver un nombre positif qui donne dans ce polynôme un résultat
positif ainsi que tout nombre plus grand. Or l'inégalité $Ex>0$ est évidemment 
satisfaite par toute solution positive de l'inégalité 

\begin{equation}
x^m-Nx^{m-n}-Nx^{m-n-1}-\dots-Nx-N > 0
\end{equation}

à savoir $x^m-N\frac{x^{m-n+1}-1}{x-1} > 0$. Cette inégalité n'étant pas satisfaite en général
par $(1)$, la marche que nous suivons

\end{document}
